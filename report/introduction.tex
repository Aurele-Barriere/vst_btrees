% Main-Memory Database Systems
  With memory sizes increasing and prices dropping, the assumption that most of the values of a database system must reside on disk no longer holds.
  This resulted in the emergence of several \textit{Main Memory Database Systems}~\cite{mmdb}, where most (or all) of the data is in the memory.
  As a result, file I/O is no longer a bottleneck for such systems.\todo{(there's more than just avoiding i/o)}
  Examples include MassTree~\cite{masstree}, VoltDB, Hekaton, SAP HANA, MemSQL and others.\todo{(add citations)}.

% MassTree
  MassTree~\cite{masstree} is an example of such Main-Memory Database System.
  It stores values, indexed by keys, directly in the memory.
  It uses a combination of the well-studied \btrees\ (a variation of the BTrees introduced in~\cite{btrees}) and Tries data structures.
  \todo{describe results and context}

% DeepDB
  We \todo{(we? deepdb?)} decided to implement and verify a high-performance Main-Memory Key-Value Store, inspired by MassTree.
  Just like MassTree, our implementation should use B+Trees inside a Trie-like structure.
  However, our B+Tree Implementation uses Cursors to reduce the complexity of some operations for partially-sorted data.

% VST Verification
  Formal verification of C programs has been the focus of many recent works.
  In particular, CompCert~\cite{compcert,compcert2} is a fully verified C compiler in Coq~\cite{coq}, which formally defines C and assembly semantics and proves that the source and compiled programs have equivalent behaviors.
  This allow for program verification at the source level, as the compiled program is guaranteed to run as specified by the source code.
  Then, the Verified Software Toolchain (VST)~\cite{vst} allows you to write specifications for C programs and formally verify in Coq that they are respected (using the same C semantics as CompCert).
  \todo{more about VST}

% Contribution
  This paper focuses on the implementation, specification and verification of the B+Trees with Cursors Library. We first chose to consider a sequential implementation.
  Ultimately, our data structure should allow a concurrent usage, but we believe that formal verification of a sequential program is a mandatory first step towards the verification of a concurrent one.\todo{badly put}
  Given a formal specification of an abstract key-value data structure with cursors and a first version of a B+Tree with cursors implementation, the work presented here consisted in rewriting the C code to comply with the specification, then prove it correct using VST.

% Results
  \todo{results?}
  The verification of the B+Tree Library has allowed us to find several bugs in the original implementation.

% Outline
  We first present the B+Tree Structure with cursors and its implementation.
  We then define an equivalent formal model, that is used for the verification.
  We then present the VST verification of our implementation.
  Finally, we prove that the formal model (proved to be equivalent to the C program), complies with the abstract specification.\todo{not done yet}
  % add section references

TODO: I need a section (related works?) where I present the work of Brian and Tosin. couls also present related VST proofs.
