% Main-Memory Database Systems
  With memory sizes increasing and prices dropping, the assumption that most of the values of a database system must reside on disk no longer holds.
  This resulted in the emergence of several \textit{Main Memory Database Systems}~\cite{mmdb}, where most (or all) of the data is in the memory.
  As a result, file I/O is no longer a bottleneck for such systems.
  Examples include MassTree~\cite{masstree}, VoltDB~\cite{voltdb}, MemSQL~\cite{memsql}, Hekaton, SAP HANA and others.

% MassTree
  MassTree~\cite{masstree} is an example of such Main-Memory Database System.
  It stores values, indexed by keys, directly in the memory.
  It uses a combination of the well-studied \btrees\ (a variation of the BTrees introduced in~\cite{btrees}) and Tries data structures.
  MassTree's performance is similar to MemCached, and better than VoltDB, MongoDB and Redis, other high-performance storage systems.

% VST Verification
  Formal verification of C programs has been the focus of many recent works.
  In particular, CompCert~\cite{compcert,compcert2} is a fully verified C compiler in Coq~\cite{coq}, which formally defines C and assembly semantics and proves that the source and compiled programs have equivalent behaviors.
  This allows for program verification at the source level, as the compiled program is guaranteed to run as specified by the source code.
  Then, the Verified Software Toolchain (VST)~\cite{vst} allows you to write specifications for C programs and formally verify in Coq that they are respected (using the same C semantics as CompCert).
  VST is itself proved sound in Coq with regards to CompCert's semantics.
  VST has been used to prove correctness of many C programs, including cryptographic primitives such as SHA-256~\cite{sha} or OpenSSL's HMAC~\cite{hmac}.

% Contribution
  This paper focuses on the implementation, specification and verification of a B+Trees with Cursors Library.
  As of today, this library only deals with sequential operations.
  Ideally, our data structure should allow a concurrent usage, but we believe that formal verification of a sequential program is a mandatory first step towards the verification of a concurrent one.
  Given a formal specification of an abstract key-value data structure with cursors and a first version of a B+Tree with cursors implementation, the work presented here consisted in rewriting the C code to comply with the specification, then prove it correct using VST.

% Results
  \todo{results?}
  The verification of the B+Tree Library has allowed us to find several bugs in the original implementation (see Section~\ref{sec:bugs}).

% Outline
  We first present the B+Tree Structure with cursors and its implementation (Section~\ref{sec:btrees})
  We then define an equivalent formal model, that is used for the verification (Section~\ref{sec:model}).
  We then present in Section~\ref{sec:verif} the VST verification of our implementation.
  We present in Section~\ref{sec:bugs} some of the bugs we encountered and fixed while verifying the library.
