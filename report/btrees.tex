\subsection{B+Trees}
% Context
The first part of our work consisted in modifying a \btrees\ with cursors library.
% Properties
%\btrees\ are high-performance in-memory data structures.
\btrees\ are ordered and self-balanced.
This allows for fast access to the data (located at the leaves), as it suffices to go down the tree using the keys in the nodes to find the next one.
\btrees\ have been well studied~\cite{dbms} and implemented numerous times. % They are used in \todo{SQL,SQLite,...}.
A \btree\ example can be seen \textbf{Fig.}~\ref{bt}.
The keys in the leaves point to a record (indicated in the figure by a * next to the key).
The fanout value in this example is 4 (the maximum number of keys in each node).
Every internal node has $n+1$ children, where $n$ is its number of keys.
Traditionally, \btrees\ implement cross-links between leaves, meaning that each leaf node points to the previous and the next one.
This allows for range queries as one can always find the next record.
However, because our implementation uses cursors (see below), these links are not needed.

% Operations
The available operations on a \btree\ typically include inserting a record (associated with a key), deleting a record, accessing the record associated to a given key (if it has been inserted) and accessing the records corresponding to a range of keys.
We briefly describe some of these operations. More details can be found in~\cite{dbms}.
\paragraph{Insertion} Takes a key and a record.
If the key already exists in the \btree, its associated record should be updated to the given one. Otherwise, the record is added to the leaves.
If it is inserted in a full leaf node, this node should be split into two. Then, the middle key is copied into the parent node to point to the new one.
When inserting this new key, it is possible that the parent node was also full.
In that case, we keep on recursively splitting nodes until some parent can accept a new entry, or the root has to be split (thus creating a new level in the tree).
This algorithm keeps the tree balanced.
The complexity of inserting a new record is $\mathcal{O}(\log_{f}(n))$, where $f$ is the fanout of the tree and $n$ the number of records.
Indeed, the number of operations (key comparisons and node splittings) is linear in the depth of the tree.

For instance, a record with the key 4 should be inserted in \textbf{Fig.}~\ref{bt} in the first leaf node (containing keys 0, 1, 2 and 3).
Because this node is full, it should be split into two: one containing 0, 1, 2, and the other containing keys 3 and 4.
Then, key 2 should be inserted into the parent node (with keys 5, 9, 12, 15).
Because this node is full too, it should be split.
Because internal nodes do not contain records, the middle key can simply be pushed to the root without being copied.
The resulting \btree, after insertion, can be seen \textbf{Fig}~\ref{btinsert}.

\beforeinsert
\afterinsert

\paragraph{Accessing a Record} Takes a key as input. Because the tree is ordered, it suffices to go down from the root.
At each node, the values of the keys indicate which child to go to at the next step.
The complexity is $\mathcal{O}(\log_{f}(n))$, where $f$ is the fanout of the tree and $n$ the number of records.

\subsection{Cursors}
% Why are cursors useful
On numerous occasions, inserting or accessing data can be done on partially sorted keys.
In this case, the operations will target and affect the same part of the \btree.
But this locality isn't exploited, as the functions always start from the root.
Cursors aim to exploit the locality of operations on close keys, by remembering the last position where the tree was modified or accessed.
Then for instance, to look for a new key, the function can start from the last accessed leaf. If the searched key is in the same node, then the function has constant time complexity.
Otherwise, it should go up to the node's parent before going down again. If the searched key is close, this should significantly reduce the number of node accesses.

% Cursor definition
A cursor is implemented as an array of pairs. Each pair contains a pointer to a node and an index for that node.
The first node of the cursor should be the root of the \btree.
Then the following nodes describe a path from the root to an entry in a leaf node.
An example of cursor can be found in \textbf{Fig.}~\ref{cursor}.
A cursor's length should always be equal to the depth of the \btree\ it refers to, thus pointing to a record.
To the best of our knowledge, cursors of this kind have been used only in the \btrees\ implementation of SQLite~\cite{sqlite}.

\cursor

% how are the functions changed
With a cursor, basic functions (insertion, accessing a record) do not start at the root anymore, but from the leaf node that the cursor is pointing to.
For instance, when accessing the record for a given key $k$, if $k$ is in the same leaf node as the cursor, then we can fetch the associated record in constant-time.
Whenever $k$ is not in the same leaf node, we need to go up in the \btree\ using the previous levels of the cursors, until we reach a parent node of the desired key.
We can then go down the \btree\ until $k$ is found.
For instance, if we search for key 14 in \textbf{Fig.}~\ref{cursor}, we can first see that 14 is not in the leaf node pointed to by the cursor (containing only keys from 5 to 8).
We then go up in the \btree, using the cursor, to the internal node containing 5, 9, 12 and 15.
Because it contains keys less and greater than 14, we know that this node is a parent node for 14.
Finally, we go down to the fourth child, containing keys 12 and 14, and we can get the desired record.
This new algorithm is still $\mathcal{O}(\log_{f}(n))$ in the worst-case (if we need to go up to the root of the tree, then go down to the leaves),
but is faster when looking for close keys.
In particular, accessing the record of each key sequentially has amortized constant time complexity~\cite{tosin}.

Inserting a new record has a similar complexity.
To perform range queries, we need to introduce a new function, \textit{movetonext}, that moves the cursor to the next record in a \btree.

\subsection{Implementation}
We modified DeepDB's implementation of \btrees\ with cursors. The C types are the following:

\begin{multicols}{2}
\lstinputlisting{types.c}
\end{multicols}

A node contains an array of entries.
Each of these entries contains the pointer to the associated child (for internal nodes) or the associated record (for leaf nodes).
Because internal nodes need one more child than they have entries, the node also contains \texttt{ptr0} which holds the pointer to the first child (or \texttt{NULL} for a leaf node).
Nodes also contain booleans that indicate if the node is the first or the last of its level in the \btree.
This is used to speed up some functions.
Finally, a cursor has an array of ancestors (the nodes, from the root to the leaf node pointed to) and an array of indexes locating the next child at each level.

This library contains 37 functions, all of which are verified in VST (see Section~\ref{sec:verif}).\todo{all?}
Deletion hasn't been implemented yet.
The functions that can be used by a client of the library include:
\begin{itemize}
\item \lstinline{Relation_T RL_NewRelation(void);}  creates a new, empty relation.
\item \lstinline{Cursor_T RL_NewCursor(Relation_T relation);}  creates a cursor for a given relation, pointing to the first key.
\item \lstinline{Bool RL_CursorIsValid(Cursor_T cursor);}  returns true if the cursor is pointing to a key.
\item \lstinline{Key RL_GetKey(Cursor_T cursor);}  returns the key pointed to by a cursor.
\item \lstinline{const void* RL_GetRecord(Cursor_T cursor);}  returns the record pointed to by a cursor.
\item \lstinline{void RL_PutRecord(Cursor_T cursor, Key key, const void* record);}  inserts a new record.
\item \lstinline{Bool RL_MoveToKey(Cursor_T cursor, Key key);}  moves the cursor to a given key.
\item \lstinline{Bool RL_MoveToFirst(Cursor_T btCursor);}  moves the cursor to the first key.
\item \lstinline{void RL_MoveToNext(Cursor_T btCursor);}  moves the cursor to the next position.
\item \lstinline{void RL_MoveToPrevious(Cursor_T btCursor);}  moves the cursor to the previous position.
\item \lstinline{Bool RL_IsEmpty(Cursor_T btCursor);}  returns true if the \btree\ of the cursor is empty.
\item \lstinline{size_t RL_NumRecords(Cursor_T btCursor);}  returns the number of keys in the \btree.
\end{itemize}


