% contextualization
In order to prove the correctness of our \btrees\ library, we first need a formal model, in Coq, for each type and function.
This model will be used to specify each C function with VST.
Then, the correctness can be proved by showing that the functions of the formal model comply with the formal specification, thus leveraging the proof to a Coq one without C semantics.

% why not Brian's model ?
DeepDB already contained a formal model that complied with the abstract relation axiomatization~\cite{brian}.
However, this \btree\ model does not exactly mimic the behavior of the C code.
For instance, entries of the C implementation contain a pointer to the child with greater keys.
Then, the node contains a pointer to the first child.
However, in this formal model, each entry contains a node which keys are lesser, and the node holds the pointer to the last child.
Moreover, the \texttt{First} and \texttt{Last} booleans used to speed up the functions are not present in each node, and not used in the functions.
As a result, we decided to define another formal model, as close as possible to the C code.
This new formal model will be used to specify the C functions with VST.


% Types
\begin{figure}
\begin{lstlisting}[language=Coq]
(* Btree Types *)
Inductive entry (X:Type): Type :=
     | keyval: key -> V -> X -> entry X
     | keychild: key -> node X -> entry X
with node (X:Type): Type :=
     | btnode: option (node X) -> listentry X -> bool -> bool -> bool -> X -> node X
with listentry (X:Type): Type :=
     | nil: listentry X
     | cons: entry X -> listentry X -> listentry X.

Definition cursor (X:Type): Type := list (node X * index). (* ancestors and index *)
Definition relation (X:Type): Type := node X * X.  (* root and address *)
\end{lstlisting}
\label{coqtypes}
\caption{Coq Formal Model Types}
\end{figure}

% Types explanation
\paragraph{Types}\textbf{Fig.}~\ref{coqtypes} presents the Coq Types for \btrees, cursors and relations. \todo{update for dependent relation type}
We can see that entries have a key, and either a record (of type \texttt{V}) or a child (of type \texttt{node X}).
Nodes have three booleans, representing the \texttt{isLeaf}, \texttt{First} and \texttt{Last} of the C code.
The $ptr_0$ of a node is represented with an option, as Leaf Nodes don't have any.
These types are parametrized by a type \texttt{X}, that can be either \texttt{val} or \texttt{unit}.
Using \texttt{val} allows to add the C address of each node, record or relation directly into the formal type.
Usually in VST proofs, the address of a structure is not part of the formal model, but given as an argument to a representation function.
Indeed, one might want to hide from the formal specifications everything that is not part of an abstract structure.
However, because cursors hold pointers to the nodes, their representation function must know where each node of a relation is in the memory.
These locations cannot be abstracted without losing the possibility of representing several trees and cursors in the memory.
The addresses can then be included directly into each node by using \texttt{node val}, and an user can still reason about abstract \btrees\ with \texttt{btnode unit}.

A cursor is implemented in Coq as a list of nodes and indexes.
This corresponds to the arrays found in the C code. The list starts at the root and its head is the current node and index.
Its length indicates the cursor's depth.
Finally, a relation is simply a root (node) and the address at which the representation is in the memory.% The depth and number of records can be easily computed.

\paragraph{Functions} Then, each C function must have an equivalent in the formal Coq model.
For instance, \textbf{Fig.}~\ref{movetofirst} presents the coq \texttt{moveToFirst} function.
It takes as input the next node to go down to. If this node is a Leaf Node, then it returns the cursor with the new node, and the index 0 (pointing to the first record).
Otherwise, it goes down $ptr_0$, and adds to the cursor the next node and the index \texttt{im} (representing -1).

\begin{figure}
\begin{lstlisting}[language=Coq]
(* takes a PARTIAL cursor, n next node (pointed to by the cursor)
   and goes down to the first key *)
Fixpoint moveToFirst {X:Type} (n:node X) (c:cursor X) (level:nat): cursor X :=
  match n with
    btnode ptr0 le isLeaf First Last x =>
    match isLeaf with
    | true => (n,ip 0)::c
    | false => match ptr0 with
               | None => c      (* not possible, isLeaf is false *)
               | Some n' => moveToFirst n' ((n,im)::c) (level+1)
               end
    end
  end.
\end{lstlisting}
\caption{MoveToFirst in the formal model}
\label{movetofirst}
\end{figure}  
