% contextualization
In order to prove the correctness of our \btrees\ library, we first need a formal model, in Coq, of each type and function.
This model will be used to specify each C function with VST.
Then, the correctness can be proved by showing that the functions of the formal model comply with the formal specification, thus leveraging the proof to a Coq one without C semantics.

% why not Brian's model ?
We already had a formal model that complied with the abstract relation axiomatization (\todo{ref to the section where this is presented}).
However, this \btree\ model does not exactly mimic the behavior of the C code.
For instance, the \texttt{First} and \texttt{Last} booleans used to speed up the functions are not present in each node, and not used in the functions.
Of course, these values could be computed, and we can prove that the optimized algorithm is equivalent to the classical one, but this would make the proof more difficult. \todo{other examples}.
As a result, we decided to split the proof. We first prove the correctness of the C code with regards to another formal model, as close as possible to the C code.
Then, we discuss the equivalence of the two formal models in section~\ref{sec:spec}.

\paragraph{Types}
% Types
\begin{figure}
\begin{lstlisting}
(* Btree Types *)
Inductive entry (X:Type): Type :=
     | keyval: key -> V -> X -> entry X
     | keychild: key -> node X -> entry X
with node (X:Type): Type :=
     | btnode: option (node X) -> listentry X -> bool -> bool -> bool -> X -> node X
with listentry (X:Type): Type :=
     | nil: listentry X
     | cons: entry X -> listentry X -> listentry X.

Definition cursor (X:Type): Type := list (node X * index). (* ancestors and index *)
Definition relation (X:Type): Type := node X * X.  (* root and address *)
\end{lstlisting}
\label{fig:coqtypes}
\caption{Coq Formal Model Types}
\end{figure}

% Types explanation
\textbf{Fig.}~\ref{fig:coqtypes} presents the Coq Types for \btrees, cursors and relations.
These types are parametrized by a type \texttt{X}, that can be either \texttt{val} or \texttt{unit}.
Using \texttt{val} allows to add the C address of each node, record or relation directly into the formal type.
We can see that entries have a key, and either a record (typed \texttt{V}) or a child (typed \texttt{node X}).
Nodes have three booleans, representing the \texttt{isLeaf}, \texttt{First} and \texttt{Last} of the C code.
The $ptr_0$ of a node is represented with an option, as Leaf Nodes don't have any.

A cursor is implemented in Coq as a list of nodes and indexes.
This corresponds to the arrays found in the C code. The list starts at the root and its head is the current node and index.
Its length indicates the cursor's depth.
Finally, a relation is simply a root (node) and the address at which the representation is in the memory.% The depth and number of records can be easily computed.

\begin{figure}
\begin{lstlisting}
(* takes a PARTIAL cursor, n next node (pointed to by the cursor)
   and goes down to the first key *)
Fixpoint moveToFirst {X:Type} (n:node X) (c:cursor X) (level:nat): cursor X :=
  match n with
    btnode ptr0 le isLeaf First Last x =>
    match isLeaf with
    | true => (n,ip 0)::c
    | false => match ptr0 with
               | None => c      (* not possible, isLeaf is false *)
               | Some n' => moveToFirst n' ((n,im)::c) (level+1)
               end
    end
  end.
\end{lstlisting}
\label{fig:movetofirst}
\caption{MoveToFirst in the formal model}
\end{figure}  

\paragraph{Functions} Then, each C function must have an equivalent in the formal Coq model.
For instance, \textbf{Fig.}~\ref{fig:movetofirst} presents the coq \texttt{moveToFirst} function.
It takes as input the next node to go down to. If this node is a Leaf Node, then it returns the cursor with the new node, and the index 0 (pointing to the first record).
Otherwise, it goes down $ptr_0$, and adds to the cursor the next node and the index \texttt{im} (representing -1).
