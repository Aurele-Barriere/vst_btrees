\subsection{Using the Verified Software Toolchain}
% VC
The Verified Software Toolchain uses Verifiable C to prove correctness of C programs.
Verifiable C consists of a language and a program logic.
The language of Verifiable C is a subset of CompCert's Clight~\cite{clight}.
Clight was introduced as an intermediate language in CompCert, where all expressions are pure and side-effect free.
The program logic of Verifiable C is a higher-order Separation Logic~\cite{sep} (an extension of Hoare Logic~\cite{hoare}).
VST includes many tools, proved theorems and Coq tactics to assist the user in writing a forward separation logic proof in Coq that relates Clight's semantics (as defined in CompCert) and a formal specification.

% work flow 
When proving the correctness of a C program, a VST user should first generate an equivalent Clight program.
CompCert includes the \texttt{clightgen} tool to do so. Then, the user should write a specification in Coq for each C function.
Finally, the user can prove the correctness of each function separately.
%The Coq goal starts with a Separation Triple (see below).

\subsubsection{Verifiable C's Separation Logic.}
% Hoare
Hoare Logic has been extensively studied and used.
The correctness of a program is represented by a Hoare triple $\{P\}~c~\{Q\}$, where $P$ and $Q$ are formulas (respectively called precondition and postcondiction), and $c$ is a program.
Hoare Logic's simple rules allow to derive Hoare triples. For instance, the composition rule states that, if $\{P\}~c_1~\{Q\}$ and $\{Q\}~c_2~\{R\}$ hold, then $\{P\}~c_1;c_2~\{R\}$ holds.

% SEP
Separation Logic is an extension of Hoare Logic. Correctness is still modeled with a triple $\{P\}~c~\{Q\}$. However, the formulas for preconditions and postconditions are augmented with a new operator $*$.
Informally, $P_1~*~P_2$ means that the heap (or memory) can be split into two disjoint parts, one where $P_1$ holds, and another where $P_2$ holds.
This operator is convenient when dealing with multiples objects in the memory.
For instance, if $\btrep(n,p)$ means that the node $n$ is represented in the memory at address $p$, then $\btrep(n_1,p_1)*\btrep(n_2,p_2)$ means that the two nodes are in the heap, at different addresses (in particular, $p_1\neq p_2$).
Without this operator, one would have to add many propositions of the form $p_1\neq p_2$ when describing multiple objects in the heap, making the proof harder.
Separation Logic also defines the \textit{magic wand} operator $\wand$. Informally, $P~\wand~Q$ means that if the heap is extended with a disjoint part where $P$ holds, then $Q$ holds on the total heap.
This is particularly useful to extract information from a separation construct, as seen in Section~\ref{subsec:btverif}.

\def\prop{\lstinline[language=Coq]{PROP}}
\def\local{\lstinline[language=Coq]{LOCAL}}
\def\sep{\lstinline[language=Coq]{SEP}}

\subsubsection{Function Specification in Verifiable C.}
In Verifiable C, a precondition or postcondition formula consists in three sets: \prop, \local\ and \sep.
\prop\ contains assertions of type \texttt{Prop} in Coq.
\local\ binds local variables to values. For instance, one could write \texttt{temp \_a (Vint(Int.repr 0))} to state that local variable \texttt{a} is bound to 0.
\sep\ contains spatial assertions in separation logic. Writing \texttt{P;Q} in \sep\ means that $\texttt{P}~*~\texttt{Q}$ holds.
\textbf{Fig.}~\ref{spec} shows one of our specifications.
The precondition contains several requirements, like \texttt{next\_node c (get\_root r) = Some n}, meaning that \texttt{n} is the node pointed to by the partial cursor.
Comparing the \sep\ clauses of the precondition and postcondition allows to understand what happens in the memory.
Here, the relation is unchanged, while the cursor is modified to represent the one returned by the Coq function \texttt{moveToFirst} (see \textbf{Fig.}~\ref{movetofirst}).

\begin{figure}
\begin{lstlisting}[language=Coq]
Definition moveToFirst_spec : ident * funspec :=
  DECLARE _moveToFirst
  WITH r:relation val, c:cursor val, pc:val, n:node val
  PRE[ _node OF tptr tbtnode, _cursor OF tptr tcursor, _level OF tint ]
    PROP(partial_cursor c r; next_node c (get_root r) = Some n)
    LOCAL(temp _cursor pc; temp _node (getval n);
          temp _level (Vint(Int.repr(Zlength c))))
    SEP(relation_rep r; cursor_rep c r pc)
  POST[ tvoid ]
    PROP()
    LOCAL()
    SEP(relation_rep r; cursor_rep (moveToFirst n c (length c)) r pc).
\end{lstlisting}
\caption{Formal Specification of the \texttt{moveToFirst} function}
\label{spec}
\end{figure}


\subsection{Augmented Types}
\label{subsec:rep}
For each formal type, we need a representation predicate to specify how it is represented in the memory.
For instance, in the VST verification of Binary-Search Trees~\cite{vst}, the predicate \lstinline[language=Coq]{tree_rep (t: tree val) (p: val) : mpred} is defined.
Such predicates relate a formal structure to a statement on the content of the memory.
However, this statement (of type \texttt{mpred}) holds more information than the formal model.

For instance, consider the following linked list in a C program:

\begin{center}
  \begin{tikzpicture}[list/.style={rectangle split, rectangle split parts=2,
        draw, rectangle split horizontal}, >=stealth]
    
    \node[list] (A) {$a_0$ \nodepart{two} $q_0$};
    \node[list] (B) [right=of A] {$a_1$ \nodepart{two} $q_1$};
    \node[list] (C) [right=of B] {$a_2$ \nodepart{two} \texttt{nullval}};
    \draw[*->,xshift=1cm] let \p1 = (A.east), \p2 = (A.center) in (\x1,\y2) -- (B);
    \draw[*->] let \p1 = (B.east), \p2 = (B.center) in (\x1,\y2) -- (C);
  \end{tikzpicture}
\end{center}

This list is a representation of the formal list $[a_0,a_1,a_2]$.
Each element contains a pointer to the following one.
The pointers $q_0, q_1$ should not appear in the formal model but are needed by the representation predicate.
One should write \lstinline{listrep} such that this additional representation information is hidden from the user.

The usual way to do that in VST consists in existentially quantifying over these values.
For instance, one could write:
\begin{lstlisting}[language=Coq]
Fixpoint list_rep (l:list A) (p:val) : mpred :=
  match l with
    | [ ] => emp
    | ai::l' => EX pi:val, cell_rep ai pi p * list_rep l' pi
  end.
\end{lstlisting}

Where \lstinline{cell_rep ai pi p} means that there exists a cell in the memory containing value \texttt{ai} and pointer \texttt{pi} at address \texttt{p}.
\lstinline{list_rep l p} describes that list \texttt{l} is represented in the memory, starting at address \texttt{p}. This is convenient for many data-structures.
However, if an external data-structure holds pointers to the cells, we would want the value of such pointers to be the same as the ones quantified over.
For instance, if some structure contains a pointer to the cell containing $a_1$, we need the pointer to be equal to $q_0$.
But with the previous definition of \lstinline{list_rep}, $q_0$ isn't known outside of the quantifier's scope.

We suggest using an augmented type for lists that holds both the formal model and the additional representation information.
We first define the type \lstinline{concrete_list} as
\lstinline[language=Coq]{Definition concrete_list A : Type := list (A * val).}
We then write an erasure function of type \lstinline[language=Coq]{concrete_list A -> list A} (here the function \lstinline[language=Coq]{map fst}).
We can change \lstinline[language=Coq]{list_rep} as follows:
\begin{lstlisting}[language=Coq]
Fixpoint list_rep (l:concrete_list A) (p:val) : mpred :=
  match l with
    | [ ] => emp
    | (ai,pi)::l' => cell_rep ai pi p * list_rep l' pi
  end.
\end{lstlisting}

Finally, if a structure holds a pointer to a cell, its formal model is given an augmented cell (an element of the augmented list, of type \lstinline{A * val}) and the pointer's representation can use the \lstinline{val}, which is guaranteed to be the address of the cell in the memory.
% This allows to represent structures that point to another one.

In our \btrees\ library, we have cursors containing pointers to subnodes of a \btree. We thus define a general \btree\ type, as seen on \textbf{Fig.}~\ref{coqtypes}, parametrized with a type \texttt{X}.
Finally we define the following types:\\
\lstinline[language=Coq]{Definition concrete_tree A : Type := node val.}\\
\lstinline[language=Coq]{Definition abstract_tree A : Type := node unit.}

We then define a representation lemma on augmented trees and augmented cursors (in the appendix, \textbf{Fig.}~\ref{btnoderep}).
We also prove multiple lemmas about these representations. For instance, if a node is represented in the memory at some address $p$, then $p$ is a valid pointer.

\subsection{\btrees\ Verification}
\label{subsec:btverif}
% rep functions
% locals
We define properties of the \btrees\ and cursors that are required to prove correctness.
Such definitions include \texttt{partial\_cursor c r}, meaning that \texttt{c} is a correct cursor for relation \texttt{r} that stops at an internal node.
By correct, we mean that each node is the $n^{th}$ child of the previous node, where $n$ is the previous index.

% subnode, subnode_rep
We then prove multiple lemmas to help us reason with the memory representations of each structure.
For instance, the function \texttt{currNode} returns a pointer to the last node of a cursor.
In the verification proof, we need to access this pointer by proving that the current node is represented somewhere in the memory.
However, the function precondition only states that the root node is represented in the memory.
To solve this, we first need to prove that the current node is a subnode of the root.
This is done by proving the following theorem:
\begin{lstlisting}[language=Coq]
  forall X (c:cursor X) r,
  complete_cursor_correct_rel c r -> subnode (currNode c r) (get_root r)
\end{lstlisting}

We then need to prove that, if a node $root$ is represented in the heap, and some other node $n$ is a subnode of $root$, then $n$ is also represented in the heap.
This is true because the function \btrep\ calls itself recursively on each children (see \textbf{Fig.}~\ref{btnoderep}).
However, because \btrep\ is a separating clause, we can't simply use an implication. We need to rewrite it as a separation conjunction.
This is done by proving the following theorem, \textbf{subnode\_rep}:

\begin{lstlisting}[language=Coq]
  forall n root, subnode n root ->
  btnode_rep root = btnode_rep n * (btnode_rep n -* btnode_rep root)
\end{lstlisting}

\todo{explanation? explain wandframeelim? wouldn't it be too technical?}

% body proofs
After proving many such theorems and writing every function specification (see \textbf{Fig.}~\ref{spec}), we need to prove the correctness of each function.
These proofs consists in proving a Separation triple $\{P\}~c~\{Q\}$, where $P$ and $Q$ are the precondition and postcondition defined in the specification, and $c$ is the Clight function.
Each triple is proved by moving forward through $c$.
For instance, if the first instruction of $c$ is an assignment \texttt{a:=0}, using the \lstinline[language=Coq]{forward} tactic of VST will turn the proof goal $\{P\}~\texttt{a=0};~c'~\{Q\}$ into $\{P'\}~c'~\{Q\}$,
where $P'$ is $P$ with the new \local\ binding: \texttt{temp \_a (Vint(Int.repr 0))}.
Occasionally, other goals have to be proven.
Every time an array is accessed, we must prove that the index is in the right range.
Every time a pointer is dereferenced, we must prove that it is valid.
Even if VST is able to infer the next precondition to use for sequences of instructions, we must still provide loop invariants when going through a loop or a branching statement.
When calling another function of the \btrees\ library, we must prove that the precondition of this function holds.
Finally, when we went through every statement of $c$, we are left with a goal $\{P\}~\texttt{skip}~\{Q\}$, where $Q$ is the function postcondition, and $P$ is the new precondition obtained after moving through the program.
Such goals (called entailments) are proved using separation logic rules implemented in VST.

\subsection{Results}
\todo{TODO: what has been proved? what's left?}
In the appendix, Section~\ref{subsec:proof} presents the VST proof of the function creating a new node.
