\subsection{\btree\ insertion}
\afterinsert

\subsection{Moving a cursor to a new key}
\label{subsec:bug}
\bugcursor
In the \btree\ \textbf{Fig.}~\ref{bugcursor}, moving the cursor to key 4 should make it point to the end of the first leaf node.
Then, in the original implementation, accessing the record pointed to by the cursor would return the one associated with key 3.
According to the abstract specification, the returned value should be the one associated with the next key (here, 5).

\newpage
\subsection{\btrees\ with cursors library}

\begin{figure}
  The functions that can be used by a client of the library are:
  \begin{itemize}
  \item \lstinline{Relation_T RL_NewRelation(void);}  creates a new, empty relation.
  \item \lstinline{Cursor_T RL_NewCursor(Relation_T relation);}  creates a cursor for a given relation, pointing to the first key.
  \item \lstinline{Bool RL_CursorIsValid(Cursor_T cursor);}  returns true if the cursor is pointing to a key.
  \item \lstinline{Key RL_GetKey(Cursor_T cursor);}  returns the key pointed to by a cursor.
  \item \lstinline{const void* RL_GetRecord(Cursor_T cursor);}  returns the record pointed to by a cursor.
  \item \lstinline{void RL_PutRecord(Cursor_T cursor, Key key, const void* record);}  inserts a new record.
  \item \lstinline{Bool RL_MoveToKey(Cursor_T cursor, Key key);}  moves the cursor to a given key.
  \item \lstinline{Bool RL_MoveToFirst(Cursor_T btCursor);}  moves the cursor to the first key.
  \item \lstinline{void RL_MoveToNext(Cursor_T btCursor);}  moves the cursor to the next position.
  \item \lstinline{void RL_MoveToPrevious(Cursor_T btCursor);}  moves the cursor to the previous position.
  \item \lstinline{Bool RL_IsEmpty(Cursor_T btCursor);}  returns true if the \btree\ of the cursor is empty.
  \item \lstinline{size_t RL_NumRecords(Cursor_T btCursor);}  returns the number of keys in the \btree.
  \end{itemize}
  \caption{\btrees\ Library Functions}
  \label{header}
\end{figure}
\newpage
\subsection{Representation of \btrees\ in Separation Logic}

\begin{figure}
\begin{lstlisting}[language=Coq]
Fixpoint entry_rep (e:entry val): mpred:=
  match e with
  | keychild _ n => btnode_rep n
  | keyval _ v x => value_rep v x
  end
with btnode_rep (n:node val):mpred :=
  match n with btnode ptr0 le b First Last pn =>
  EX ent_end:list(val * (val + val)),
  malloc_token Tsh tbtnode pn *
  data_at Tsh tbtnode (Val.of_bool b,(
                       Val.of_bool First,(
                       Val.of_bool Last,(
                       Vint(Int.repr (Z.of_nat (numKeys n))),(
                       match ptr0 with
                       | None => nullval
                       | Some n' => getval n'
                       end,(
                       le_to_list le ++ ent_end)))))) pn *
  match ptr0 with
  | None => emp
  | Some n' => btnode_rep n'
  end *
  le_iter_sepcon le
  end
with le_iter_sepcon (le:listentry val):mpred :=
  match le with
  | nil => emp
  | cons e le' => entry_rep e * le_iter_sepcon le'
  end.
\end{lstlisting}
\caption{Representing a \btree\ node in the memory}
\label{btnoderep}
\end{figure}

\newpage
\subsection{A VST proof}
\label{subsec:proof}
\begin{figure}
  \begin{lstlisting}
static BtNode* createNewNode(Bool isLeaf, Bool First, Bool Last) {
    BtNode* newNode;
    newNode = (BtNode*) surely_malloc(sizeof (BtNode));
    if (newNode == NULL) {
      return NULL;
    }
    newNode->numKeys = 0;
    newNode->isLeaf = isLeaf;
    newNode->First = First;
    newNode->Last = Last;
    newNode->ptr0 = NULL;
    return newNode;
}
  \end{lstlisting}
  \caption{The C code for creating a new node}
\end{figure}

\begin{figure}
  \begin{lstlisting}[language=Coq]
Definition empty_node (b:bool) (F:bool) (L:bool) (p:val):node val := (btnode val) None (nil val) b F L p.
    
Definition createNewNode_spec : ident * funspec :=
  DECLARE _createNewNode
  WITH isLeaf:bool, First:bool, Last:bool
  PRE [ _isLeaf OF tint, _First OF tint, _Last OF tint ]
    PROP ()
    LOCAL (temp _isLeaf (Val.of_bool isLeaf);
         temp _First (Val.of_bool First);
         temp _Last (Val.of_bool Last))
    SEP ()
  POST [ tptr tbtnode ]
    EX p:val, PROP ()
    LOCAL (temp ret_temp p)
    SEP (btnode_rep (empty_node isLeaf First Last p)).
  \end{lstlisting}
  \caption{Creating a new node in the formal model and specification of the C function}
\end{figure}

\begin{figure}
  \begin{lstlisting}[language=Coq]
Lemma body_createNewNode: semax_body Vprog Gprog f_createNewNode createNewNode_spec.
Proof.
  start_function.
  forward_call (tbtnode).       (* t'1=malloc(sizeof tbtnode) *)
  - split. simpl. rep_omega.
    split; auto.
  - Intros vret.
    forward_if (PROP (vret<>nullval)
    LOCAL (temp _newNode vret; temp _isLeaf (Val.of_bool isLeaf);
    temp _First (Val.of_bool First); temp _Last (Val.of_bool Last))
    SEP (malloc_token Tsh tbtnode vret * data_at_ Tsh tbtnode vret)).
    + forward.                  (* return null *)
    + forward. entailer!.
    + Intros. 
      forward.                  (* newNode->numKeys = 0 *)
      unfold default_val. simpl.
      forward.                  (* newnode->isLeaf=isLeaf *)
      forward.                  (* newnode->First=First *)
      forward.                  (* newnode->Last=Last *)
      forward.                  (* newnode->ptr0=null *)
      forward.                  (* return newnode *)
      Exists vret. entailer!.
      Exists (list_repeat Fanout (Vundef, (inl Vundef):(val+val))).
      simpl. cancel.
      apply derives_refl.
Qed.
  \end{lstlisting}
  \caption{Proving the function correct with VST}
\end{figure}

    

    
    
